%\documentclass[draft, 12pt]{book}
\documentclass[12pt]{book}

\usepackage[utf8]{inputenc}
\usepackage[english,russian]{babel}
\usepackage[colorlinks]{hyperref}

\title{Руководство пользователя Git}
\date{Начало перевода: 30 сентября 2010 г.\linebreak Последняя правка: \today}

\pagestyle{headings}

\begin{document}
\maketitle
\tableofcontents
\frenchspacing
%%%%%%%%%%%%%%%%%%%%%%%%%%%%%%%%%%%%%%%%%%%%%%%%%%%%%%%%%%%%%%%%%%%%%%%%%%%%%%%%
%
% FRONT MATTER
% ВВЕДЕНИЕ
%
%%%%%%%%%%%%%%%%%%%%%%%%%%%%%%%%%%%%%%%%%%%%%%%%%%%%%%%%%%%%%%%%%%%%%%%%%%%%%%%%
\frontmatter
\hypertarget{preface}{}\chapter{Введение}
Git~--- это быстрая распределённая система контроля версиями.

Это руководство рассчитано на тех, кто обладает базовыми знаниями командной
строки Unix, и совершенно не разбирается в git.

\href{#chapter1}{\emph{Глава~1. Репозитории и ветки}} и 
\href{#chapter2}{\emph{Глава~2. Исследование истории git}} объясняют как 
получить и изучить проект использующий git. Прочитав эти главы вы поймёте как 
создавать и тестировать отдельные версии программного проекта, восстанавливать 
предыдущие состояния и т. д. Тех, кто занимается разработкой, заинтересуют главы
\href{#chapter3}{\emph{Глава~3.\linebreak Разработка вместе с git}} и 
\href{#chapter4}{\emph{Глава~4. Совместная разработка}}. После\-дующие главы 
охватывают более специализированные вопросы.

Наиболее полная документация доступна на страницах \verb|man| и при вызове
команды \href{http://www.kernel.org/pub/software/scm/git/docs/git-help.html}
{git-help(1)}. Например, для комманды \linebreak \verb|git clone <repo>|, вы 
можете использовать:

\begin{verbatim}
$ man git-clone
\end{verbatim}

или

\begin{verbatim}
$ git help clone
\end{verbatim}

Так же вы можете использовать любую программу просмотра \linebreak руководства
\verb|man|.
Для подробной информации обратитесь к 
\href{http://www.kernel.org/pub/software/scm/git/docs/git-help.html}
{git-help(1)}.

Ещё прочтите \href{#appendixa}{Приложение~А. Быстрый обзор git} для получения
\linebreak общего представления о командах Git без подробных разъяснений.

%!!!!! этот абзац нужно будет пересмотреть после перевода приложения Б
И наконец, \href{#appendixb}{Приложение~Б. Заметки и список ссылок для этого
\linebreak руководства} поможет вам в желании сделать это руководство более
\linebreak полным.
%%%%%%%%%%%%%%%%%%%%%%%%%%%%%%%%%%%%%%%%%%%%%%%%%%%%%%%%%%%%%%%%%%%%%%%%%%%%%%%%
%
% END
%
%%%%%%%%%%%%%%%%%%%%%%%%%%%%%%%%%%%%%%%%%%%%%%%%%%%%%%%%%%%%%%%%%%%%%%%%%%%%%%%%
\end{document}
