%\documentclass[a4paper, 12pt, oneside, draft]{scrreprt}
\documentclass[a4paper, 12pt, oneside]{scrreprt}

\usepackage[T2A]{fontenc}
\usepackage[utf8]{inputenc}
\usepackage{color}
\usepackage[colorlinks, citecolor=blue, pagecolor=magenta, linkcolor=blue, unicode]{hyperref}
\usepackage[english, russian]{babel}
\usepackage{misccorr}
\usepackage{listings}

\title{Руководство пользователя Git}
\date{Начало перевода: 30 сентября 2010 г.\linebreak Последняя правка: \today}

\pagestyle{headings}

\begin{document}

\lstset{language=bash, 
		frame=trbl,
		frameround=tttt,
		basicstyle=\tt , 
		keywordstyle=\textbf,		
		commentstyle=\textit
		}

\maketitle
\tableofcontents


%%%%%%%%%%%%%%%%%%%%%%%%%%%%%%%%%%%%%%%%%%%%%%%%%%%%%%%%%%%%%%%%%%%%%%%%%%%%%%%%
%
% ВВЕДЕНИЕ
%
%%%%%%%%%%%%%%%%%%%%%%%%%%%%%%%%%%%%%%%%%%%%%%%%%%%%%%%%%%%%%%%%%%%%%%%%%%%%%%%%
\newpage
\section*{Введение}
\hypertarget{#preface}{}

Git "--- это быстрая распределённая система контроля версиями.

Это руководство рассчитано на тех, кто обладает базовыми знаниями командной
строки Unix, и совершенно не разбирается в git.

\href{#chapter1}{Глава~1. Репозитории и ветки} и 
\href{#chapter2}{Глава~2. Исследование истории git} объясняют как 
получить и изучить проект использующий git. Прочитав эти главы вы поймёте как 
создавать и тестировать отдельные версии программного проекта, восстанавливать 
предыдущие состояния и т. д. Тех, кто занимается разработкой, заинтересуют главы
\href{#chapter3}{Глава~3. Разработка вместе с git} и 
\href{#chapter4}{Глава~4. Совместная разработка}. Последующие главы 
охватывают более специализированные вопросы.

Наиболее полная документация доступна на страницах \emph{man} и при вызове
команды \href{http://www.kernel.org/pub/software/scm/git/docs/git-help.html}
{git-help(1)}. Например, для комманды \emph{git clone <repo>}, вы 
можете использовать:

\begin{lstlisting}
$ man git-clone
\end{lstlisting}

или

\begin{lstlisting}
$ git help clone
\end{lstlisting}

Так же вы можете использовать любую программу просмотра руководства
\emph{man}.
Для подробной информации обратитесь к 
\href{http://www.kernel.org/pub/software/scm/git/docs/git-help.html}
{git-help(1)}.

Ещё прочтите \href{#appendixa}{Приложение~А. Быстрый обзор git} для получения
общего представления о командах Git без подробных разъяснений.

%!!!!! этот абзац нужно будет пересмотреть после перевода приложения Б
И наконец, \href{#appendixb}{Приложение~Б. Заметки и список ссылок для этого
руководства} поможет вам в желании сделать это руководство более
полным.
%%%%%%%%%%%%%%%%%%%%%%%%%%%%%%%%%%%%%%%%%%%%%%%%%%%%%%%%%%%%%%%%%%%%%%%%%%%%%%%%
%
% CHAPTER 1
%
%%%%%%%%%%%%%%%%%%%%%%%%%%%%%%%%%%%%%%%%%%%%%%%%%%%%%%%%%%%%%%%%%%%%%%%%%%%%%%%%
\chapter{Репозитории и ветки}
\hypertarget{#chapter1}{}

\section{Как получить репозиторий}

Раз уж вы читаете это руководство, то будет полезно скачать репозиторий git для 
экспериментов.

Лучшим способом сделать это будет использование команды
\href{http://www.kernel.org/pub/software/scm/git/docs/git-clone.html}{git-clone(1)}
для получения копии существующего репозитория. Если у вас ещё нет проектов на 
примете, то вот пара примеров:

\begin{lstlisting}
# git itself (approx. 10MB download):
$ git clone git://git.kernel.org/pub/scm/git/git.git
# the Linux kernel (approx. 150MB download):
$ git clone git://git.kernel.org/pub/scm/linux/kernel/git
                                  /torvalds/linux-2.6.git
\end{lstlisting}

Создание копии репозитория может занять много времени для больших проектов, но 
только в первый раз.

Команда клонирования создаёт новую директорию с именем проекта (<<git>> и <<linux-2.6>>
для вышеприведённых примеров. В созданной папке вы увидите файлы проекта, называемые
\href{#def_working_tree}{рабочим деревом (working tree)} вместе со специальной папкой 
<<.git>>, которая содержит всю информацию об истории проекта.



\section{Как создать новую версию проекта}

Git лучше рассмаривать как инструмент для хранения истории набора колекции файлов.
Он хранит истории сжатой, в виде набора взаимосвязанных состояний содержимого
проекта. Каждое такое состояние называется \href{#def_commit}{коммитом (commit)}.

Эти состояния не обязательно идут друг за другом, от старого к новому. Вместо этого
они могут развиваться параллельно, образуюя \href{#def_branch}{ветки (branch)},
которые могут сливаться и разделяться.

Один репозиторий git может отслеживать развитие нескольких ветвей. Он реализует это
введением списка \href{#def_head}{заголовков (heads)}, которые ссылаются на
последние коммиты в каждой ветви. Команда \href{http://www.kernel.org/pub/software/scm/git/docs/git-branch.html}{git-branch(1)}
покажет вам список заголовков ветвей:

\begin{lstlisting}
$ git branch
* master
\end{lstlisting}

Недавно склонированный репозиторий содержит единственную ветвь, называему по умолчанию
<<master>>.
% дописать

Большинство проектов ещё используют \href{#def_tag}{теги (tags)}. Теги, подобно
заголовкам, ссылаются на историю проекта. Их можно просмотреть с помощью команды
\href{http://www.kernel.org/pub/software/scm/git/docs/git-tag.html}{git-tag(1)}:

\begin{lstlisting}
$ git tag -l
v2.6.11
v2.6.11-tree
v2.6.12
v2.6.12-rc2
v2.6.12-rc3
v2.6.12-rc4
v2.6.12-rc5
v2.6.12-rc6
v2.6.13
...
\end{lstlisting}

Теги всегда указывают на одну и ту же версию проекта, в то время как заголовки
модифицируются вместе с развитием проекта.

Для создания заголовка указывающего на новую ветку одной из версий проекта
осуществляется с помощью команды
\href{http://www.kernel.org/pub/software/scm/git/docs/git-checkout.html}{git-checkout(1)}:

\begin{lstlisting}
$ git checkout -b new v.2.6.13
\end{lstlisting}

Рабочая директория содержащая проект была помещена тегом \emph{v.2.6.13}. Теперь
команда \href{http://www.kernel.org/pub/software/scm/git/docs/git-branch.html}{git-branch(1)}
покажет две ветки. Текущая ветвь помечена <<звёздочкой>>:

\begin{lstlisting}
$ git branch
  master
* new
\end{lstlisting}


%%%%%%%%%%%%%%%%%%%%%%%%%%%%%%%%%%%%%%%%%%%%%%%%%%%%%%%%%%%%%%%%%%%%%%%%%%%%%%%%
%
% END
%
%%%%%%%%%%%%%%%%%%%%%%%%%%%%%%%%%%%%%%%%%%%%%%%%%%%%%%%%%%%%%%%%%%%%%%%%%%%%%%%%
\end{document}
